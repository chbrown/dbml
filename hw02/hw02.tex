\documentclass[11pt]{report}
\newcommand{\titl}{Machine Learning: Homework 2}
\usepackage[pdftex]{graphicx}
\usepackage{henrian-more}

\begin{document}

\section{Data generation}

There are two variables we can plot accuracy on: the top N eigenvectors, and the number K of training examples.

Also, k in kNN, while taking the most common out of 20 K, for example.

What happens if you trim the vector down and only use the top n eigenvectors? How accurately can you classify using nearest neighbor using cosine similarity? Plot figures showing the accuracy as the number of training points increases, and as the number of eigenvectors increases.

Must the images be centered exactly? what if they aren't?

\end{document}


